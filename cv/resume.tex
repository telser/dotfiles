\documentclass[margintitle,line]{res}
\usepackage[a4paper,left=1.0cm,right=4.5cm,top=1.0cm,bottom=1.0cm]{geometry}
\usepackage[utf8]{inputenc}
\usepackage[colorlinks=true,urlcolor=black]{hyperref}

% specify some fonts and colors
\renewcommand{\familydefault}{\sfdefault}
\renewcommand{\sectionfont}{\scshape}
\renewcommand{\namefont}{\LARGE \bfseries}
\renewcommand{\titlefont}{\bf}
\renewcommand{\datesfont}{\bf}
\definecolor{linecolor}{RGB}{25,25,112}

% override defaults
\renewcommand{\employerfont}{}

% ugly kludge: we really have to define proper subsections in the .cls file
\renewcommand{\subsection}[1]{\section{\normalfont #1}}

\setlength{\parskip}{1ex}

\hypersetup{
  pdfauthor   = {Trevis Elser},
  pdftitle    = {Curriculum Vitae: Trevis Elser},
  pdfsubject  = {Curriculum Vitae},
  pdfkeywords = {Trevis Elser, Curriculum Vitae, CV, resume},
}

\begin{document}
\name{Trevis Elser}

\begin{resume}

% Specify the format of work entries
\begin{format}
\dates{l}\\
\title{l}\employer{r}\\
\body\\
\end{format}

\section{Contact Information}

678-439-7517  \hfill {Code:}
\href{https://gitlab.com/u/telser}{https://gitlab.com/u/telser}  \\
\href{mailto:trevis.elser@clickscape.com}{\nolinkurl{trevis.elser@clickscape.com}} \hfill {Other Code:} \href{https://github.com/telser}{https://github.com/telser} \\
Based in metro Atlanta, GA
%\hfill {Web:} \url{http://www.silencedpoets.com/} \\

\section{Education}

\title{(Unfinished) Computer Science}
\employer{\llap{Emory University}}
\dates{August 2012 -- November 2013}
\begin{position}
I focused on algorithms and discrete mathematics. Specifically, graph
algorithms and network theory applications to software. I left before
finishing to take a break from academia.
\end{position}

\title{B.Sc. Computer Science}
\employer{Southern Polytechnic State University \\ GPA 3.68}
\dates{January 2008 -- August 2012}
\begin{position}
I focused on functional programming, using Haskell for projects when
permitted. My senior project was on using artificial intelligence to
improve testing of table top games. Further, I completed a project for
converting medical codes.
\end{position}

\title{B.Sc. Mathematics}
\employer{Southern Polytechnic State University \\ GPA 3.68 }
\dates{January 2008 -- August 2012}
\begin{position}
I concurrently completed a seperate degree in Mathematics. My senior
project covered predictive analytics specifically in financial
mathematics. Specifically comparing sinusoidal volatility in the
Black-Scholes equation with Generalized Autoregressive Conditional
Heteroskedacity.
\end{position}

\section{Work \ \ Experience}
\ \\
\subsection{Remunerated}

\title{Software Engineer}
\employer{\llap{Clickscape}}
\dates{December 2013 -- Present}
\begin{position}
  Polyglot software development in Clojure, Erlang, Haskell and Javascript.
  Part of a small team where resposiblities for new
  features, maintence, testing, and deployment are all shared. The
  main product is an event driven system, with RESTful services to
  an Angular display. An older Ruby on Rails product was discontinued.
  A new product that supplants the Ruby on Rails as well as exceeding it in scope
  is being written in Erlang. \\ \ \\
  Personal highlights include championing the use of Haskell, modifying core
  architecture of the Clojure product to provide deep reporting with the Riemann
  event system, development of Leiningen plugins for codebase analysis and reporting
  as well as setup/implementation of JIRA.\\ \ \\
 Programming languages employed include Clojure, Erlang, Haskell, Javascript,
 and Ruby. \\ \ \\
 Technology highlights include: Amazon tools (EC2, S3, Opsworks, etc.),
 Angular, Cabal, Core.Typed, Elastic Search, Git, JIRA, Leiningen,
 PostgreSQL, RabbitMQ, Riemann, Selenium, SQLite3, Zotonic.
\end{position}

\title{Teaching Assistant}
\employer{\llap{Emory University}}
\dates{August 2012 -- November 2013}
\begin{position}
 I was a teaching assistant with main responsibilities including
 grading, office hours, leading review sessions, and exam
 proctoring. I was the TA for CS 224 Mathematical Foundations of
 Computer Science twice and CS 323 Data Structures and Algorithms
 once. Taking a break from academia meant a break from this position.
\end{position}

\title{Software Engineering Intern}
\employer{\llap{Clickscape}}
\dates{May 2013 -- August 2013}
\begin{position}
 Work extending an existing Ruby on Rails application with new
 features and assisting in initial design and implementation of a
 complementary system in  Clojure. Additionally, worked on
 standardizing development and production environments using Chef,
 Opsworks and Vagrant. \\ \ \\
 Programming languages employed included Clojure, Javascript and Ruby. \\ \ \\
 Technologies used included Chef, Git, MySQL, Opsworks, SQLite,
 Ruby on Rails, and Vagrant.
\end{position}

\title{Teaching Assistant}
\employer{\llap{Southern Polytechnic State University}}
\dates{May 2012 -- August 2012}
\begin{position}
 As the assistant for a course on data structures, I was responsible
 for helping students, grading labs, creation of test code for labs,
 and improvements to the structure and content of labs.
\end{position}

\title{Tutor}
\employer{\llap{Southern Polytechnic State University}}
\dates{August 2011 -- August 2012}
\begin{position}
I was a tutor for the School of Computing and Software Engineering
covering introductory programming courses in Java, C++ and C\#, data
structures, and algorithms.
\end{position}

%
\subsection{Voluntary}

\title{Contributor}
\employer{\llap{Miscellaneous}}
\dates{Various times}
\begin{position}
I have also made various contributions to the following projects:
\begin{itemize}
\item{Oh-My-ZSH, a collection of plugins for ZSH- the Z Shell}
\item{Paktahn, a Common Lisp package management helper for Arch Linux}
\end{itemize}
\end{position}

\title{Maintainer}
\employer{\llap{Miscellaneous}}
\dates{Various times}
\begin{position}
Projects that I maintain, or started can generally be found on my
Gitlab page. These include:
\begin{itemize}
\item{clj-twilio, a Clojure wrapper for the Twilio REST api}
\item{dep-graph, a plugin for Leiningen that creates the dependency
    graph of a project}
\item{lein-report, a plugin for Leiningen that creates a report about
    a project based on other Leiningen plugins.}
\end{itemize}
\end{position}

%%%%%%%%%%%%%%%%%%%%%%%%%%%%%%%%%%%%%%%%

\section{Interests and Skills}

I have strong interests in data science, functional programming,
software engineering, programming languages, and BSD/Linux software development.

Programming Languages (intermediate): Clojure, Haskell

Programming Languages (novice): C, Javascript, Python, Erlang

Programming Languages (interested in): Common Lisp, Elm, Forth,
Go, OCaml, R, Rust

Markup Languages: LaTeX, Markdown

Operating systems: FreeBSD, various Linux distributions, Mac OS X, with a personal
preference for FreeBSD and/or Debian Linux.

Other techonolgy tools I use/prefer include: Cabal, Emacs, Git, Leiningen, Mosh, ZSH

Outside of computers, I spend time with music in various
capacities. My personal music usually can be categorized as electronic
and experimental. Several projects have been released through
different channels, both with label support and on my own.

\section{References}

Available on request.

\end{resume}
\end{document}
