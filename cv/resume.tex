\documentclass[margintitle,line]{res}
\usepackage[a4paper,left=1.0cm,right=4.5cm,top=1.0cm,bottom=1.0cm]{geometry}
\usepackage[utf8]{inputenc}
\usepackage[colorlinks=true,urlcolor=blue]{hyperref}

% specify some fonts and colors
\renewcommand{\familydefault}{\sfdefault}
\renewcommand{\sectionfont}{\scshape}
\renewcommand{\namefont}{\LARGE \bfseries}
\renewcommand{\titlefont}{\bf}
\renewcommand{\datesfont}{\bf}
\definecolor{linecolor}{RGB}{25,25,112}

% override defaults
\renewcommand{\employerfont}{}

% ugly kludge: we really have to define proper subsections in the .cls file
\renewcommand{\subsection}[1]{\section{\normalfont #1}}

\setlength{\parskip}{1ex}

\hypersetup{
  pdfauthor   = {Trevis Elser},
  pdftitle    = {Curriculum Vitae: Trevis Elser},
  pdfsubject  = {Curriculum Vitae},
  pdfkeywords = {Trevis Elser, Curriculum Vitae, CV, resume},
}

\begin{document}
\name{Trevis Elser}

\begin{resume}

% Specify the format of work entries
\begin{format}
\title{l}\employer{r}\\
\dates{l}\\
\body\\
\end{format}

\section{Contact Information}

678-439-7517  \hfill {Github Code:} \href{https://github.com/telser}{https://github.com/telser}  \\
\href{mailto:trevis.elser@gmail.com}{\nolinkurl{trevis.elser@gmail.com}} \hfill {Gitlab Code:}
\href{https://gitlab.com/u/telser}{https://gitlab.com/u/telser} \\
Based in Metro Atlanta, GA
%\hfill {Web:} \url{http://www.silencedpoets.com/} \\

\section{Education}

\title{B.Sc. Computer Science}
\employer{Southern Polytechnic State University \\ GPA 3.68}
\dates{January 2008 -- August 2012}
\begin{position}
Focus on functional programming, using Haskell for projects when
permitted. My senior project was on using artificial intelligence to
improve testing of table top games. Further, I completed a project for
converting medical codes.
\end{position}

\title{B.Sc. Mathematics}
\employer{Southern Polytechnic State University \\ GPA 3.68 }
\dates{January 2008 -- August 2012}
\begin{position}
I concurrently completed a separate degree in Mathematics. My senior
project covered predictive analytics specifically in financial
mathematics. Specifically comparing sinusoidal volatility in the
Black-Scholes equation with Generalized Autoregressive Conditional
Heteroskedacity, an expectation-maximization algorithm. I gave a presentation
on this at the MAA regional conference for 2012.
\end{position}

\title{(Unfinished) Ph.D Computer Science}
\employer{\llap{Emory University}}
\dates{August 2012 -- November 2013}
\begin{position}
I focused on algorithms and discrete mathematics. Specifically, graph
algorithms and network theory applications to software. I left before
finishing to take a break from academia.
\end{position}

\section{Work \ \ Experience}
\ \\
\subsection{Remunerated}

\title{Software Engineer}
\employer{\llap{Manheim}}
\dates{March 2016 -- Present}
\begin{position}
  Programming Languages used:
  Ruby, Scala, Typescript

  Technology highlights include:
  Angular2, AWS, Elasticsearch, MySQL, Rails, Sinatra
\end{position}

\title{Senior Software Engineer}
\employer{\llap{CoinX Inc.}}
\dates{October 2015 -- January 2016}
\begin{position}
  Sole developer on a money transaction system working directly with the
  compliance staff. Responsibilities include greenfield
  development, maintence, and testing of a web service written in Haskell. Tasked
  with ensuring systems developed meet regulatory standards including PCI. A
  previous system written in Python was decomissioned due to non-compliance and
  the Haskell system meant to replace it with compliance ensured from the outset.

  % other tech highlights
  Technology highlights include: AWS, Amazon Linux, Git, Haskell, PostgreSQL,
  Purescript, RDS, Servant
\end{position}

\title{Software Engineer}
\employer{\llap{Clickscape/Village Realty}}
\dates{December 2013 -- November 2015}
\begin{position}
  Village Realty is a premium real estate agency with a custom home search site
  at clickscape.com that uses technology to provide better customer engagement
  and service during the home buying or home selling process.

  I was a developer reporting directly to the CTO and CEO with responsibilities
  for greenfield development, new features, maintenance, testing, and deployment.

  % Primary dev
  Primary development was of three systems that interacted with each other.
  A public facing website  written in Erlang with traffic on the order of 50k
  unique visitors per month. A visitor tracking system, for internal use written
  in Clojure. As well as a listing management tool with an invitation only
  interface, semi-public, written in Haskell. A variety of technologies were
  chosen for what best fit the requirements of the component, but much of the
  Clojure codebase was deprecated with gradual feature replacement by the other
  systems. The small development team meant that I worked on all three systems.

  %% Overall three
  %% main components of a distributed system were developed with the entire
  %% team working on all three. These components were written in Haskell,
  %% Erlang and Clojure. Each had their own web interface, with the Erlang
  %% component being public facing and the other two internal only. A variety
  %% of technologies were chosen for what best the requirements of the component,
  %% but the Clojure component was deprecated with gradual feature replacement
  %% by the Haskell component.

  %% Main development is a distributed system with components written in Erlang,
  %% Haskell, and Clojure along with two separate interfaces, both web based, one
  %% using Angular and the other based on Django-templates, served via Erlang,
  %% with JavaScript as necessary (no framework).

% Personal highlights
  Several projects are personal highlights for me. I had a leadership role
  in several infrastructure projects, such as developing continuous integration
  and deployment mechanisms which saved the team time on a weekly or daily basis
  as prior all of those tasks were manual. Further, I scripted server changes
  and updates, with reports, to remove more manual interaction. Moving the
  servers to FreeBSD from a mixture of Amazon Linux and Ubuntu increased
  consistency and led to a reduction in resource utilization. Those resource
  drops and a general re-architecting of the AWS infrastructure let me
  reduce our total infrastructure bill by more than 30 percent. Also, I was
  given a leadership role in changing the code-base to use PostgreSQL from Cassandra.
  Even further, I was assigned all interviewing, managing and university paperwork
  for a student intern.

  %% Personal highlights include:
  %% Being given a leadership role in a switch from Cassandra to PostgreSQL.
  %% Moving all servers to FreeBSD from a mix of Amazon Linux and Ubuntu.
  %% Interviewing, managing, and being the University liaison for a student intern.
  %% Development of continuous integration and deployment mechanisms.
  %% Re-architecting AWS server and service usage resulting in spending reduction.

  % Secondary dev
  I had other general development responsibilities as well. I wrote an
  internal data analysis tool written in Clojure. Several small data visualization
  tools or reports in Haskell, interacting with 3rd-party services and internal
  data only, were also among the projects I worked on outside of the ``core''
  development product(s).

  %% Other development responsibilities given to me included A tool generating reports based on internal and external data sources
  %% written in Haskell.
  %% Automated server auditing scripted via Python.
  %% An internal data analysis tool written in Clojure.

  % Tech transitions
  Several technology transitions occurred during my tenure for different components.
  Some examples of systems and/or technologies changed include:
  \begin{itemize}
    \item{Moving from a Ruby on Rails component to a custom one written in Erlang.}
    \item{Replacing many semi-independent Clojure components with Haskell
      as part of a general phase out.}
    \item{Using PostgreSQL in place of both Cassandra and MySQL}
    \item{Using Elasticsearch in place of Sphinx, and adding it as a data layer
      in addition to the primary datastore}
  \end{itemize}

  % languages
  Programming languages employed include:
  Clojure, Erlang, Haskell, JavaScript, Python, Ruby, and Shell.

  % other tech highlights
  Technology highlights include: Amazon tools (EC2, S3, Cloudformation, etc.),
  Angular, Ansible, Elasticsearch, FreeBSD, Git, PostgreSQL, RabbitMQ, Riemann,
  Selenium, SQLite3, Yesod.
\end{position}

\title{Teaching Assistant}
\employer{\llap{Emory University}}
\dates{August 2012 -- November 2013}
\begin{position}
 I was a teaching assistant with main responsibilities including
 grading, office hours, leading review sessions, and exam
 proctoring. I was the TA for CS 224 Mathematical Foundations of
 Computer Science twice and CS 323 Data Structures and Algorithms
 once. Taking a break from academia meant a break from this position.
\end{position}

\title{Software Engineering Intern}
\employer{\llap{Clickscape}}
\dates{May 2013 -- August 2013}
\begin{position}
 Work extending an existing Ruby on Rails application with new
 features and assisting in initial design and implementation of a
 complementary system in  Clojure. Additionally, worked on
 standardizing development and production environments using Chef,
 Opsworks and Vagrant.

 Programming languages employed included: Clojure, JavaScript and Ruby.

 Technologies used included: Chef, Git, MySQL, Opsworks, SQLite,
 Ruby on Rails, and Vagrant.
\end{position}

%% Don't delete old jobs, just hide them from output :)

%% \title{Teaching Assistant}
%% \employer{\llap{Southern Polytechnic State University}}
%% \dates{May 2012 -- August 2012}
%% \begin{position}
%%  As the assistant for a course on data structures, I was responsible
%%  for helping students, grading labs, creation of test code for labs,
%%  and improvements to the structure and content of labs.
%% \end{position}

%% \title{Tutor}
%% \employer{\llap{Southern Polytechnic State University}}
%% \dates{August 2011 -- August 2012}
%% \begin{position}
%% I was a tutor for the School of Computing and Software Engineering
%% covering introductory programming courses in Java, C++ and C\#, data
%% structures, and algorithms.
%% \end{position}

%
\subsection{Voluntary}

\title{Open Source Contributions}
\employer{\llap{Various Projects}}
\dates{Current}
\begin{position}
Open source work is a passion for me. Thus far I have contributed to a variety
of projects, nearly all of which can be found on my
\href{http://github.com/telser}{github}. Some highlights of this work include:
\begin{description}
\item[Purescript Libraries] \hfill \\
  \href{http://www.purescript.org/}{Purescript} is an
  open source pure functional compile to Javascript language. I have contributed
  to a number of both core and community libraries, around 50 in total.
\item[Keter] \hfill \\
  Keter is a deployment system for web applications originally
  written for \href{http://www.yesodweb.com/}{Yesod}, a Haskell web framework.
\item[Clojure Libraries and Leiningen Plugins] \hfill \\
  During my time at Clickscape I wrote and released a few Clojure projects
  including a library for the \href{https://www.twilio.com/}{Twilio} api and
  reporting plugins for the \href{http://leiningen.org/}{Leiningen} build tool.
\end{description}
\end{position}
%% \title{Maintainer}
%% \employer{\llap{Miscellaneous}}
%% \dates{Various times}
%% \begin{position}
%% Projects that I maintain(ed), or started can generally be found on my
%% Gitlab page. These include:
%% \begin{itemize}
%% \item{clj-twilio, a Clojure wrapper for the Twilio REST api}
%% \item{dep-graph, a plugin for Leiningen that creates the dependency
%%     graph of a project}
%% \item{lein-report, a plugin for Leiningen that creates a report about
%%   a project based on other Leiningen plugins.}
%% \item{riemann-hs, a Haskell library for sending events to a Riemann
%%   server (see riemann.io)}
%% \end{itemize}
%% \end{position}

%% \title{Contributor}
%% \employer{\llap{Miscellaneous}}
%% \dates{Various times}
%% \begin{position}
%% I have also made various contributions to the following projects:
%% \begin{itemize}
%% \item{Keter, a deployment system for web applications originally for Yesod}
%% \item{Oh-My-ZSH, a collection of plugins for ZSH- the Z Shell}
%% \end{itemize}
%% \end{position}


%%%%%%%%%%%%%%%%%%%%%%%%%%%%%%%%%%%%%%%%

\section{Interests and Skills}
\ \\
\subsection{Professional}

I have strong interests in data science, functional programming,
software engineering, programming languages, and BSD/Linux software development.

Programming Languages (most comfortable): Clojure, Haskell, Purescript

Programming Languages (some experience): Erlang, JavaScript, Ruby

Programming Languages (minor experience): C, Elm, Python, R, Scala, Typescript

Programming Languages (interested in, but have not used): Factor, Forth, Frege, Go, Julia, OCaml,
OpenCL, Prolog, Rust

Markup Languages: LaTeX, Markdown, Org-mode

Operating systems: FreeBSD, Linux (various distributions), Mac OS X, OpenBSD and
Solaris, with a personal preference for FreeBSD.

Other technology tools I use and/or prefer include: Emacs, Git, Mosh, ZSH
\subsection{Hobby}

Outside of work, I spend time with music in various
capacities. My personal music usually can be categorized as electronic
and experimental. Niche and hobby operating systems also have a certain appeal
to me and I've used several such as AmigaOS, BeOS, Haiku, Plan 9, and QNX.
Mathematics, specifically network theory and more generally applied graph
theory, is another personal interest.

\section{References}

Available on request.

\end{resume}
\end{document}
